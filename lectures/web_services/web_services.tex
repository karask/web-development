  
\documentclass{beamer}
\usepackage{listings}
\usepackage{graphicx}
\usepackage{epstopdf}
\usepackage{hyperref}

\lstset{
		tabsize=4,
        basicstyle=\scriptsize,
        %upquote=true,
        aboveskip={1.5\baselineskip},
        columns=fixed,
        showstringspaces=false,
        extendedchars=true,
        breaklines=true,
        prebreak = \raisebox{0ex}[0ex][0ex]{\ensuremath{\hookleftarrow}},
		frame=tRBl,
		%frameround=tttf,
		numbers=left,
		numberstyle=\tiny,
		numbersep=5pt,
        showtabs=false,
        showspaces=false,
        showstringspaces=false,
        identifierstyle=\ttfamily,
        keywordstyle=\color[rgb]{0,0,1},
        commentstyle=\color[rgb]{0.133,0.545,0.133},
        stringstyle=\color[rgb]{0.627,0.126,0.941},
        aboveskip=5pt
}

\usetheme{AnnArbor}
\usecolortheme{beaver}
\setbeamertemplate{note page}[plain]
\begin{document}

\title{Web Services}
\subtitle{example using Sinatra}
\author[Konstantinos Karasavvas]{Konstantinos Karasavvas} %\\{\small Software Architect and Engineer}}

%http://www.linkedin.com/pub/konstantinos-karasavvas/14/64b/14b
\institute{CITY College}
\date{\today} 

\begin{frame}
  \titlepage
\end{frame}

\begin{frame}
\setcounter{tocdepth}{1}
\frametitle{Table of contents}
\tableofcontents
\end{frame} 





\section{Web Services} 
\begin{frame}\frametitle{Web Services} 

  \begin{quote}
``Web Service; a software system designed to support interoperable machine-to-machine interaction over a network. [...]''
  \end{quote}\par\raggedleft--- \textup{W3C}   


  \begin{itemize}
    \item SOAP 
    \begin{itemize}
      \item protocol specification for exchanging structured information
    \end{itemize}

    \item HTTP
    \begin{itemize}
      \item application protocol
    \end{itemize}

    \item REST
    \begin{itemize}
      \item architectural style -- HTTP with constraints
    \end{itemize}


  \end{itemize}
\end{frame}




\section{SOAP} 
\begin{frame}[fragile]\frametitle{SOAP} 

  \begin{itemize}
    \item SOAP
    \begin{itemize}
      \item uses XML for the message format
      \item is independent of the transport protocol (HTTP, FTP, TCP, ...)
      \item strictly defines the format of messages
    \end{itemize}
    
    \item A SOAP message contains:
    \begin{itemize}
      \item headers
      \item action
      \item data
      \item errors
    \end{itemize}
    
  \end{itemize}

\end{frame}



\section{REST} 
\begin{frame}[fragile]\frametitle{REST} 

  \begin{itemize}
    \item Representational State Transfer
    \begin{itemize}
      \item architectural style for designing networked applications
      \item involves clients and servers sending request and responses respectively
      \item requests and responses are built around the transfer of representation of resources
    \end{itemize}
    
    \item REST recognises that:
    \begin{itemize}
      \item everything is a resource (User, Author, Book, etc.)
      \item each resource implements a standard uniform interface
      \item resources have unique name and addresses 
      \item each resource has one or more representations, and
      \item resource representations move across the network
    \end{itemize}
    
  \end{itemize}

\end{frame}



\begin{frame}[fragile]\frametitle{REST, cont.} 

  \begin{itemize}
    \item RESTful Web Service (or Web API)
    \begin{itemize}
      \item is a Web Service that is implemented using HTTP and the principles of REST
    \end{itemize}
    
    \item The action (request) is just the HTTP verb
    \item The response is just the body of the HTTP response
    \item HTTP -- CRUD
  \end{itemize}
  
  \begin{center}
    \begin{tabular}{|l|l|l|}
\hline 
GET & Read & safe and idempotent \\ 
\hline 
POST & Create (Update) & - \\ 
\hline 
PUT & Update (Create) & idempotent \\ 
\hline 
DELETE & Delete & idempotent \\ 
\hline 
HEAD & Read & safe and idempotent  \\ 
\hline 
    \end{tabular}
  \end{center}
    
\end{frame}



\section{SOAP: Example} 
\begin{frame}[fragile]\frametitle{SOAP: Example Request} 

  \begin{lstlisting}[language=bash, escapechar={^}]
GET /StockPrice HTTP/1.1
Host: example.org
Content-Type: application/soap+xml; charset=utf-8
Content-Length: nnn

<?xml version="1.0"?>
<env:Envelope xmlns:env="http://www.w3.org/2003/05/soap-envelope"
   xmlns:s="http://www.example.org/stock-service">
   <env:Body>
     <s:GetStockQuote>
          <s:TickerSymbol>IBM</s:TickerSymbol>
     </s:GetStockQuote>
   </env:Body>
</env:Envelope>
  \end{lstlisting}

\end{frame}


\begin{frame}[fragile]\frametitle{SOAP: Example Response} 

  \begin{lstlisting}[language=bash, escapechar={^}]
HTTP/1.1 200 OK
Content-Type: application/soap+xml; charset=utf-8
Content-Length: nnn

<?xml version="1.0"?>
<env:Envelope xmlns:env="http://www.w3.org/2003/05/soap-envelope"
   xmlns:s="http://www.example.org/stock-service">
   <env:Body>
     <s:GetStockQuoteResponse>
          <s:StockPrice>45.25</s:StockPrice>
     </s:GetStockQuoteResponse>
   </env:Body>
</env:Envelope>
  \end{lstlisting}
  
\end{frame}




\section{REST: Example} 
\begin{frame}[fragile]\frametitle{REST: Example Request-Response} 

  \begin{lstlisting}[language=bash, escapechar={^}]
GET /StockPrice/IBM HTTP/1.1
Host: example.org
Accept: text/xml
Accept-Charset: utf-8
  \end{lstlisting}

  \begin{lstlisting}[language=bash, escapechar={^}]
HTTP/1.1 200 OK
Content-Type: text/xml; charset=utf-8
Content-Length: nnn

<?xml version="1.0"?>
<s:Quote xmlns:s="http://example.org/stock-service">
     <s:StockPrice>45.25</s:StockPrice>
</s:Quote>
  \end{lstlisting}
  
\end{frame}



\begin{frame}[fragile]\frametitle{REST: Typical Request-Response} 

  \begin{lstlisting}[language=bash, escapechar={^}]
GET /StockPrice/IBM HTTP/1.1
Host: example.org
Accept: text/xml
Accept-Charset: utf-8
  \end{lstlisting}
  
  \begin{lstlisting}[language=bash, escapechar={^}]
HTTP/1.1 200 OK
Content-Type: text/json; charset=utf-8
Content-Length: nnn

{ 
    "Quote": {
        "StockPrice" : "45.25"
    }
}
  \end{lstlisting}

\end{frame}





\section{SOAP and REST} 
\begin{frame}[fragile]\frametitle{SOAP and REST: Comparison} 

  \begin{center}
    \begin{tabular}{|l|l|}
\hline 
\textbf{SOAP} & \textbf{REST} \\ 
\hline 
Language, platform, and \textit{transport} & Language and platform agnostic \\ 
agnostic  & \\
\hline 
Conceptually more difficult, more & Simple to develop \\ 
heavy-weight  & \\
\hline 
Verbose & Concise, no need for additional \\
 & messaging layer  \\ 
\hline 
Built-in error handling & Ad-hoc \\ 
\hline 
Message- and Transport-level & Transport-level security \\ 
security & \\
\hline 
Standards for complex distributed & Closer in design and philosophy to \\ 
environments (DTXs, etc.) & the Web \\
\hline 
    \end{tabular}   
  \end{center}

\end{frame}



\section{REST Example using Sinatra} 
\subsection{Introduction}
\begin{frame}[fragile]\frametitle{Introduction} 

  \begin{itemize}
    \item ToDoApp
    \begin{itemize}
      \item Keeps track of tasks
      \item Can create, read, update and delete a task
      \item We will use the JSON format for all responses
    \end{itemize}
  \end{itemize}

\end{frame}



\subsection{JSON}
\begin{frame}[fragile]\frametitle{JSON} 

  \begin{itemize}
    \item JavaScript Object Notation
    \begin{itemize}
      \item lightweight data-interchange format
      \item easy for humans to read and write
      \item easy for machines to parse and generate
      \item used as alternative to XML
      \item \texttt{application/json}
    \end{itemize}
  \end{itemize}
 

\end{frame}




\begin{frame}[fragile]\frametitle{JSON: Example} 
  
  \begin{lstlisting}[language=bash, escapechar={^}]
{
    "firstName": "John",
    "lastName": "Smith",
    "age": 25,
    "address": {
        "streetAddress": "21 2nd Street",
        "city": "New York",
        "postalCode": 10021
    },
    "phoneNumbers": [
        {
            "type": "home",
            "number": "212 555-1234"
        },
        {
            "type": "fax",
            "number": "646 555-4567"
        }
    ]
}
  \end{lstlisting}

\end{frame}
  




\begin{frame}[fragile]\frametitle{JSON: Example with Ruby} 
  
  \lstinputlisting[language=ruby]{code/json_example.rb}
  
  \begin{lstlisting}[language=ruby, escapechar={^}]
ruby json_example.rb 
"John"
"New York"
  \end{lstlisting}

\end{frame}
  
  
\subsection{Todo App API}
\begin{frame}[fragile]\frametitle{Todo App: Task Resource API} 

\begin{tabular}{|l|l|l|l|}
\hline 
RESOURCE & DESCRIPTION & HTTP \\
 & & CODES \\ 
\hline 
GET /tasks & Returns a JSON array of all task objects & 200 \\ 
 & {\small\texttt{[ \{ "id" : "1", }} & \\
 & {\small\texttt{    "description" : "Buy Milk" \} ...]}} & \\
\hline 
POST /tasks & Creates a new task resource; expects: & 200 \\
 & \small\texttt{\{ "description" : "Buy Milk" \}} & \\ 
 & Returns: & \\ 
 & {\small\texttt{\{ "id" : "1", }} & \\
 & {\small\texttt{    "description" : "Buy Milk" \}}} & \\
\hline 
GET /tasks/:id  & If found, returns the specified task: & 200, 404 \\ 
 & {\small\texttt{\{ "id" : "1", }} & \\
 & {\small\texttt{    "description" : "Buy Milk" \}}} & \\
\hline 
\end{tabular} 
  
\end{frame}




\begin{frame}[fragile]\frametitle{Todo App: Task Resource API} 

\begin{tabular}{|l|l|l|l|}
\hline 
RESOURCE & DESCRIPTION & HTTP \\
 & & CODES \\ 
\hline 
PUT /tasks/:id & If found, updates the task and returns & 200, 404 \\ 
 & it; expects: & \\
 & \small\texttt{\{ "description" : "Buy Shampoo" \}} & \\ 
 & Returns: & \\ 
 & {\small\texttt{\{ "id" : "1", }} & \\
 & {\small\texttt{    "description" : "Buy Shampoo" \}}} & \\
\hline 
DELETE /tasks/:id & If found, deletes the specified task & 200, 404 \\ 
 & resource & 500 \\ 
\hline 
\end{tabular} 
  
\end{frame}



\begin{frame}[fragile]\frametitle{Todo App: Gemfile, config.ru} 
  
  \lstinputlisting[language=ruby]{code/proj4/Gemfile}

  \lstinputlisting[language=ruby]{code/proj4/config.ru}
    
\end{frame}



\begin{frame}[fragile]\frametitle{Todo App: todo\_app.rb} 
  
  \lstinputlisting[language=ruby]{code/proj4/todo_app.rb}

\end{frame}




\begin{frame}[fragile]\frametitle{Todo App: models} 
  
  \lstinputlisting[language=ruby]{code/proj4/models/init.rb}

  \lstinputlisting[language=ruby]{code/proj4/models/tasks.rb}
    
\end{frame}




\begin{frame}[fragile]\frametitle{Todo App: migrations} 
  
  \lstinputlisting[language=ruby]{code/proj4/db/migrations/001_init_db.rb}
    
\end{frame}




\begin{frame}[fragile]\frametitle{Todo App: routes} 
  
  \lstinputlisting[language=ruby]{code/proj4/routes/init.rb}

\end{frame}




\begin{frame}[fragile]\frametitle{Todo App: routes, cont.} 
  
  \lstinputlisting[language=ruby, lastline=9]{code/proj4/routes/main.rb}
        
\end{frame}




\begin{frame}[fragile]\frametitle{Todo App: routes, cont.} 
  
  \lstinputlisting[language=ruby, firstline=11, lastline=19]{code/proj4/routes/main.rb}
    
\end{frame}




\begin{frame}[fragile]\frametitle{Todo App: routes, cont.} 
  
  \lstinputlisting[language=ruby, firstline=21, lastline=31]{code/proj4/routes/main.rb}
    
\end{frame}



\begin{frame}[fragile]\frametitle{Todo App: routes, cont.} 
  
  \lstinputlisting[language=ruby, firstline=33, lastline=44]{code/proj4/routes/main.rb}
    
\end{frame}




\begin{frame}[fragile]\frametitle{Todo App: routes, cont.} 
  
  \lstinputlisting[language=ruby, firstline=46]{code/proj4/routes/main.rb}
    
\end{frame}




\begin{frame}[fragile]\frametitle{Todo App: Example runs} 
  
  \begin{lstlisting}[language=ruby, escapechar={^}]
$ bundle exec rackup
  \end{lstlisting}

  \begin{lstlisting}[language=ruby, escapechar={^}]
$ curl  -X POST -d '{ "description" : "Buy milk" }' http://localhost:9292/tasks
{"id":5,"description":"Buy milk"}
  \end{lstlisting}

  \begin{lstlisting}[language=ruby, escapechar={^}]
$ curl  -X POST -d '{}' http://localhost:9292/tasks
{"id":6,"description":null}
  \end{lstlisting}

  \begin{lstlisting}[language=ruby, escapechar={^}]
$ curl  -X PUT -d '{ "description" : "Buy apples" }' http://localhost:9292/tasks/6
{"id":6,"description":"Buy apples"}
  \end{lstlisting}

  \begin{lstlisting}[language=ruby, escapechar={^}]
$ curl  -X DELETE http://localhost:9292/tasks/1
  \end{lstlisting}

  \begin{lstlisting}[language=ruby, basicstyle=\tiny, escapechar={^}]
127.0.0.1 - - [10/Aug/2013 13:53:11] "POST /tasks HTTP/1.1" 200 27 0.1428
127.0.0.1 - - [10/Aug/2013 13:54:12] "POST /tasks HTTP/1.1" 200 27 0.1544
127.0.0.1 - - [10/Aug/2013 13:55:27] "PUT /tasks/6 HTTP/1.1" 200 35 0.1576
127.0.0.1 - - [10/Aug/2013 13:57:23] "DELETE /tasks/1 HTTP/1.1" 200 - 0.1502
  \end{lstlisting}

\end{frame}



\begin{frame}[fragile]\frametitle{Todo App: Example runs, cont.} 

  \begin{lstlisting}[language=ruby, escapechar={^}]
$ curl  -X DELETE http://localhost:9292/tasks/1
  \end{lstlisting}

  \begin{lstlisting}[language=ruby, escapechar={^}]
$ curl  http://localhost:9292/tasks/5
{"id":5,"description":"Buy milk"}
  \end{lstlisting}

  \begin{lstlisting}[language=ruby, escapechar={^}]
$ curl  http://localhost:9292/tasks
[{"id":3,"description":null},{"id":4,"description":"Buy milk"},{"id":5,"description":"Buy milk"},{"id":6,"description":"Buy apples"}]
  \end{lstlisting}

  \begin{lstlisting}[language=ruby, basicstyle=\tiny, escapechar={^}]
127.0.0.1 - - [10/Aug/2013 14:01:31] "DELETE /tasks/1 HTTP/1.1" 404 - 0.0016
127.0.0.1 - - [10/Aug/2013 14:03:02] "GET /tasks/5 HTTP/1.1" 200 33 0.0017
127.0.0.1 - - [10/Aug/2013 14:04:37] "GET /tasks HTTP/1.1" 200 133 0.0018
  \end{lstlisting}

\end{frame}




\subsection{HTTP Error codes}
\begin{frame}[fragile]\frametitle{HTTP Error codes} 

  \begin{itemize}
    \item Notice that we provided the \textit{errors} as part of the response
    \begin{itemize}
      \item HTTP Status code is part of the API
      \begin{itemize}
        \item and indicates the error!
      \end{itemize}
      \item The body of the message could describe the error in more detail
    \end{itemize}

    \item An alternative is to use the body to encapsulate errors as well
    \begin{itemize}
      \item HTTP Status code is \textbf{not} part of the API
      \item The errors are described in an app-specific way
    \end{itemize}

  \end{itemize}

  \begin{lstlisting}[language=ruby, escapechar={^}]
{
    "status": "failure"
    "error" : {
        "code" : 2085
        "description" : "Exceeded user quota"
    }
}
  \end{lstlisting}

\end{frame}



\section{API Authentication}
\subsection{Introduction}
\begin{frame}[fragile]\frametitle{API Authentication} 

  \begin{itemize}
    \item HTTP Basic Authentication
    \begin{itemize}
      \item not very secure
      \item must use over secure connection: TLS or SSL
    \end{itemize}
    
    \item Typically APIs make use of an API key
    \begin{itemize}
      \item essentially a username for the remote service
      \item sign up is needed to use the API and an API key is generated 
      \item the API key needs to be passed with each request
      \item should use over secure connection: TLS or SSL
    \end{itemize}

    \item OAuth (API key/secret)
    \begin{itemize}
      \item essentially a username/password for the remote service
      \item sign up is needed to use the API and an API key/secret is generated 
      \item the API key/secret needs to be passed with each request
      \item saves creating your own key/signature system
    \end{itemize}

  \end{itemize}

\end{frame}
%http://stackoverflow.com/questions/247110/looking-for-suggestions-for-building-a-secure-rest-api-within-ruby-on-rails


\subsection{API Key Authentication}
\begin{frame}[fragile]\frametitle{API Key Authentication} 

  \begin{lstlisting}[language=ruby, escapechar={^}]
  before do
    error 401 unless valid_key?(params[:key])
  end

  helpers do
    def valid_key? (key)
      # check if key is valid!
    end
  end

  get "/" do
    {"hello" => "world"}.to_json
  end
  \end{lstlisting}

\end{frame}



\begin{frame}[fragile]\frametitle{API Key Authentication, v2 with simple extension} 

  \begin{lstlisting}[language=ruby, escapechar={^}]
  register do
    def auth (name)
      condition do
        error 401 unless send(name) == true
      end
    end
  end

  helpers do
    def valid_key?
      # check if params[:key] is valid!
    end
  end

  get "/", :auth => :valid_key? do
    {"hello" => "world"}.to_json
  end
  \end{lstlisting}

\end{frame}




\subsection{Sinatra over SSL}
\begin{frame}[fragile]\frametitle{Sinatra over SSL} 

  \lstinputlisting[language=ruby]{code/proj5/config.ru}

  \lstinputlisting[language=ruby]{code/proj5/app.rb}
  
  \begin{itemize}
    \item administrative task!
  \end{itemize}
  
\end{frame}




\end{document}